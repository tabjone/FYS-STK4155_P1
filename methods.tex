%%%%%%%%%%%%%%%%%%%%%%%%%%%%%%%%%%%%%%%%%%%%%%%%%%%%%%%%%%%%%%%%%%%%%%%%%%%%%%%%%%%%%%%%%%
\section{Method}
%%MOST IMPORTANT FOR METHOD: Put in exact parameters used to produce plots so that a reader can reproduce it. %%

%%On how we implemented: psudo-code and algos
%Maybe section about using SVD for inverting matrix, or maybe section on SVD in theory

%scaling of data. Why we did or didn't scale. Critical discussion of this.

\subsection{Datasets}
\subsubsection{Franke Function}
%What is it
%How did we implement it
%Why did we implement it
%Any testing to know its working correctly? (I think not needed)
\subsubsection{Terrain data}
%How did we implement it
%Where did we find this
%Why do we use it

\subsection{Error estimation}
%%Mean squared error
%What is it
%Why do we use it

%%R^2
%What is it
%Why do we use it

\subsection{Linear regression}

%We used the Vandermonde design matrix for a linear regression polynomial fit. No point in having a theory section about this.
\begin{center}
$V=
\begin{bmatrix} 
1 & x_{0}&x_{0}^{2}&\dots &x_{0}^{p-1}
\\1&x_{1}&x_{1}^{2}&\dots &x_{1}^{p-1}
\\1&x_{2}&x_{2}^{2}&\dots &x_{2}^{p-1}
\\ \vdots &\vdots &\vdots &\ddots &\vdots \\
1&x_{n-1}&ax_{n-1}^{2}&\dots &x_{n-1}^{p-1}
\end{bmatrix}
$
\end{center}


%%OLS
%How did we implement it
%Why did we implement it
%Did we do any testing

%%Ridge
%How did we implement it
%Why did we implement it
%Did we do any testing

%%Lasso
%How did we implement it
%Why did we implement it
%Did we do any testing

\subsection{Resampling}
%%Bootstrap
%How did we implement it
%Why did we implement it
%Did we do any testing

%%Cross-validation
%How did we implement it
%Why did we implement it
%Did we do any testing




%THIS IF FROM THE LECTURE NOTES. BUT ABIT EXTENSIVE. MIGHT NOT NEED IT
\begin{comment}
How to set up the cross-validation for Ridge and/or Lasso

* Define a range of interest for the penalty parameter.

* Divide the data set into training and test set comprising samples $\{1, \ldots, n\} \setminus i$ and $\{ i \}$, respectively.

* Fit the linear regression model by means of ridge estimation  for each $\lambda$ in the grid using the training set, and the corresponding estimate of the error variance $\boldsymbol{\sigma}_{-i}^2(\lambda)$, as

\begin{align*}
\boldsymbol{\beta}_{-i}(\lambda) & =  ( \boldsymbol{X}_{-i, \ast}^{T}
\boldsymbol{X}_{-i, \ast} + \lambda \boldsymbol{I}_{pp})^{-1}
\boldsymbol{X}_{-i, \ast}^{T} \boldsymbol{y}_{-i}
\end{align*}

* Evaluate the prediction performance of these models on the test set by $[y_i, \boldsymbol{X}_{i, \ast}; \boldsymbol{\beta}_{-i}(\lambda), \boldsymbol{\sigma}_{-i}^2(\lambda)]$. Or, by the prediction error $|y_i - \boldsymbol{X}_{i, \ast} \boldsymbol{\beta}_{-i}(\lambda)|$, the relative error, the error squared or the R2 score function.

* Repeat the first three steps  such that each sample plays the role of the test set once.

* Average the prediction performances of the test sets at each grid point of the penalty bias/parameter. It is an estimate of the prediction performance of the model corresponding to this value of the penalty parameter on novel data. It is defined as

\begin{align*}
\frac{1}{n} \sum_{i = 1}^n \log\{L[y_i, \mathbf{X}_{i, \ast}; \boldsymbol{\beta}_{-i}(\lambda), \boldsymbol{\sigma}_{-i}^2(\lambda)]\}.
\end{align*}
\end{comment}

